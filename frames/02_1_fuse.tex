%!TEX TS-program = lualatex
%!TEX encoding = UTF-8 Unicode

%\frame[plain]{ % When including a large figure or table, you don't want to have the bottom and the top of the slides.
%\frame[shrink]{ % If you want to include lots of text on a slide, use the shrink option.

\begin{frame}
    \frametitle{FUSE (File System in User Space)}
    \begin{center}
        \includegraphics[width=.7\textwidth]{frames/img/fuse_arch}
    \end{center}
    
    \begin{itemize}
        \item simplifies writing file system drivers by avoiding kernel-space code
        \item mounting / unmounting can be done as an unprivileged user
        \item file ownership can be adjusted to enable unprivileged access to the filesystem
    \end{itemize}
    
    \note{
    \begin{itemize}
        \item First explain how FS works. For now, ignore the right side
        \item syscall to enter kernel, space and user space. priviliged instructions. Device and FS drivers usually in kernel. (supervisor-mode, comparch)
        \item but what if I want to implement a FS in userspace? FUSE comes into play
        \item kernel driver exposes a queue of requests to user space...
        \item ... which is consumed by a user space daemon handling those requests
        \item Of course there is some overhead due to context switches and copying
        \item simplifies writing file system drivers by avoiding kernel-level code
        \item mounting / unmounting can be done as a unprivileged user
        \item file ownership can be adjusted to enable unprivileged access to the filesystem
    \end{itemize}
    }
\end{frame}