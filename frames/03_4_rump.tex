%!TEX TS-program = xelatex
%!TEX encoding = UTF-8 Unicode

%\frame[plain]{ % When including a large figure or table, you don't want to have the bottom and the top of the slides.
%\frame[shrink]{ % If you want to include lots of text on a slide, use the shrink option.

\frame {
    % add GNUNet logo and maybe some illustrative "surveillance-picture"
    \frametitle{RUMP Kernel}
    \begin{itemize}
        \item takes the NetBSD kernel, leaving out everything except drivers, and using those drivers as library components
        \item initial focus on driver re-use, now shifted to unikernels
        \item quite mature ecosystem (nginx, Erlang, OpenJDK...)
        \item "FS-Utils" provide mtools-like CLI for FS-access
        \item NetBSD does not support many file systems (e.g. no Ext4)
    \end{itemize}
    \note{
        \begin{itemize}
            \item NetBSD, leaving out everything except drivers
            \item using as library components
            \item no scheduler (using the host), no memory mgmt etc. => responsibility of the host
            \item initial focus on driver re-use.
            \item no shift to unikernels, i.e. link app against kernel, running in one address space
            \item docker
            \item This approach is only sustainable when the project is maintained within the same code base
        \end{itemize}
    }
\hspace{10mm}
        \begin{columns}[c]
            \begin{column}{.2\textwidth}
                \includegraphics[width=\textwidth]{frames/img/rump_logo}
            \end{column}
            \begin{column}{.2\textwidth}
                \includegraphics[width=\textwidth]{frames/img/netbsd_logo}
            \end{column}
        \end{columns}


}