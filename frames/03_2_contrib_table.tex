%!TEX TS-program = lualatex
%!TEX encoding = UTF-8 Unicode

%\frame[plain]{ % When including a large figure or table, you don't want to have the bottom and the top of the slides.
%\frame[shrink]{ % If you want to include lots of text on a slide, use the shrink option.

\begin{frame}

\frametitle{Contribution: Summary}

\begin{table}[htb]
    \centering
    
%    \renewcommand{\arraystretch}{1}
%    \rowcolors{1}{}{lightgray!40}
    \begin{tabular}{@{}m{4cm}m{1cm}m{2.5cm}m{2.2cm}@{}}
        \toprule
        \textbf{Approach}          & \textbf{OS} & \textbf{Impl. work} & \textbf{Evaluation} \\ \midrule
        Loopback-Mount             & \FAFR{}       &  -                           & detailed \\
        LKL-FUSE                   & \FAFR{}       &  bugfixes                    & detailed \\
        libguestfs-FUSE (LKL)      & \FAFR{}       &  implemented                 & detailed \\
        libguestfs-FUSE (KVM)      & \FAFR{}       &  -                           & detailed \\ \midrule
        VHD-Mount                  & \FAFR{}     &  -                           & detailed \\
        LKL-FUSE (Dokany)          & \FAFR{}     &  ported                      & detailed \\
        LKL CLI utils              & \FAFR{}     &  ported                      & detailed \\ \midrule
        various others             & -     & -                            & applicability \\
        \bottomrule
    \end{tabular}
    
\end{table}


\note{
    \begin{itemize}
        \item So, what did \textbf{I} do (besides implementing)?
        \item qualitative investigation (features) for many others (not mentioned)
        \item performance evaluation of the different solutions which I will talk about in the next section
        \item OS-native methods (loopback-mounting) were used as baseline
        
    \end{itemize}
   
}


\end{frame}
