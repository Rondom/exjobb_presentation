%!TEX TS-program = xelatex
%!TEX encoding = UTF-8 Unicode

%\frame[plain]{ % When including a large figure or table, you don't want to have the bottom and the top of the slides.
%\frame[shrink]{ % If you want to include lots of text on a slide, use the shrink option.

\begin{frame}

\frametitle{Further Work}
\begin{itemize}
    \item deeper analysis of results, testing different workloads
    \item conformance with filesystem test suites
    \item impact of mitigations for CPU vulnerabilities related to speculative execution
    \item New software features: libfuse3 writeback-caching, LKL syscall overhead improvements, Dokan 2.0, Linux Kernel FUSE improvements
    \item use case: manipulating disk images when using VMs for development
    \item use case: enhanced security in distributions for mounting untrusted filesystems
\end{itemize}

\note{
\begin{itemize}
    \item deeper analysis of results, testing different workloads
    \item conformance with filesystem test suites
    \item impact of mitigations for CPU vulnerabilities related to speculative execution
    \item New software features: libfuse3 writeback-caching, LKL syscall overhead improvements, Dokan 2.0, Linux Kernel FUSE improvements
    \item enable downloading of debug-symbols from the target using checkpointing and/or some of the technologies presented
    \item what can it be used for?
    \item use case: manipulating disk images when using VMs for development, integrate it
    \item use in production for enhanced security in distributions such as TAILS or even OpenSUSE
    \item all in all it was a great experience, learned a lot, interesting technology
\end{itemize}
}


\end{frame}
