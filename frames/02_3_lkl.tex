%!TEX TS-program = xelatex
%!TEX encoding = UTF-8 Unicode

%\frame[plain]{ % When including a large figure or table, you don't want to have the bottom and the top of the slides.
%\frame[shrink]{ % If you want to include lots of text on a slide, use the shrink option.

\begin{frame}
    \frametitle{Linux Kernel Library (LKL)}
        \begin{itemize}
            \item a library operating system
            \item "architecture" port of the Linux kernel so that the whole kernel can be run as a library
            \item does not interface with HW, but with a list of callbacks
            \item implementations for those callbacks exist for different hosts (userspace of Windows and Linux, UEFI...)
            \item only basic primitives are implemented on the host side (threading, TLS, semaphores, timers...)
            \item interface towards the "library-kernel" are standard syscalls, e.g. lkl\_sys\_getpid()
        \end{itemize}

    \note{
        \begin{itemize}
            \item \textbf{Q} library OS, an OS kernel that can be used as a library
            \item arch-port of Linux kernel. What does that mean? Kernel structured that it is easy to port to new arch such as ARM, MIPS, PPC.
            \item all code in one folder- LKL is such an architecture, but not a CPU arch in the classic sense
            \item basic primitives are implemented for each host, POSIX, Win32, Windows Kernel, Haiku OS
            \item interface is the syscalls, e.g. lkl\_sys\_getpid()
            \item CLI-utilities to exchange files and a FUSE filesystem exists
        \end{itemize}
    }
\end{frame}
