%!TEX TS-program = xelatex
%!TEX encoding = UTF-8 Unicode

%\frame[plain]{ % When including a large figure or table, you don't want to have the bottom and the top of the slides.
%\frame[shrink]{ % If you want to include lots of text on a slide, use the shrink option.

\frame[t] {
    
    \frametitle{Existing Solutions}
    \begin{columns}[t]
        \begin{column}{.45\textwidth}
            \begin{block}{Linux}
                \begin{itemize}
                    \item loopback-mounting
                    \item CLI-utilities (e.g. mtools)
                    \item alternative drivers (kernel and FUSE)
                    \item libguestfs
                    \item Linux Kernel Library (LKL)
                \end{itemize}
            \end{block}
        \end{column}
        \begin{column}{.55\textwidth}
            \begin{block}{Windows}
                \begin{itemize}
                    \item mounting VHD-files
                    
                    \item CLI-utilities (e.g ltools)
                    \item alternative drivers (kernel and FUSE)
                \end{itemize}
            \end{block}
        \end{column}
    \end{columns}
    \note{
        \begin{itemize}
            \item loopback/VHD mounting, => standard way, limited to OS FS support
            \item CLI-utilities (mtools, ltools) => non-uniform interface, re-implementation, less-proven code
            \item alternative drivers may lack the stability and feature-set of mature implementations
            \item Only libguestfs offers wide-FS support under a uniform interface. Goal: re-use OS-level code!
            \item LKL (allows using the Linux Kernel as a library)
        \end{itemize}
    }
}